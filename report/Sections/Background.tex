\section{Background}

Machine learning methods have been used extensively in the context of SDM; see \cite{vincenzi2011application}, \cite{jin2012predicting}, \cite{jackson2015citizen}, \cite{aertsen2010comparison} for examples. Beery et. al. \cite{beery2021species} provide a comprehensive review of SDM aimed at computer scientists, highlighting the common methods, terminology, and challenges associated with this area.
A common practice in previous work is to combine geospatial data for a particular species with bioclimatic variables (e.g. temperature, precipitation) and topographical attributes (e.g. elevation, slope, flow water direction). Lorena et. al. \cite{lorena2011comparing} compare the use of nine different machine learning models in predicting the distribution of thirty five Latin American plant species. Geospatial data for each species was combined with nine environmental layers, comprising of four climatic variables and five topographical attributes. Lee et. al. \cite{lee2022spatial} used a similar approach to model the potential distribution of invasive ant species under climate change. This approach seems viable for our purposes.
%The authors suggest using an ensemble method to compensate for discrepancies between models, however Hao et. al. \cite{hao2020testing} found only a slight increase in performance using an ensemble model over an individual model using a data-set of eucalyptus tree species.