\section{Introduction}
A key problem in ecology is understanding the many ecosystems of the world and how they respond to climate change, conservation efforts, and habitat destruction. Central to this work is the monitoring of species present in a fixed area and collection of data from a variety of sources. Ecologists can then perform species distribution modelling (SDM) using this data and determine the species present in an ecosystem, their respective populations, and how these populations change over time. However, the processing and analysis of data can take years due to the large number of samples taken. Machine learning offers a promising method to speed up SDM.
%, allowing for larger studies across longer timescales. 
In this report, we use various machine learning models to analyse data from iNaturalist (\url{www.inaturalist.org}), a "crowd-sourced species identification system". We evaluate our machine learning methods to determine which has the best classification accuracy for this data-set, then attempt to improve the model capabilities by training on eight new bio-climatic variables from WorldClim \cite{worldclim}, visually explored in Appendix \ref{appendix:bio}. We use the most improved model in conjunction with projected temperature data from the Met Office Hadley Centre HadGEM3-GC31-LL model \cite{met} to subsequently determine which species are most affected by climate change, which could then be used to focus conservation efforts. 