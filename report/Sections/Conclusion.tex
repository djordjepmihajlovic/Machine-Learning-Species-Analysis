\section{Conclusion}
We trained four different machine learning models on a data-set from iNaturalist containing coordinates and names of species identified at those coordinates. The four models used were a Feed-Forward Neural Network (FFNN), Random Forest (RF), Logistic Regression (LR), and K-Nearest Neighbours (KNN). We evaluated the performance of these models on the test-data by calculating the PRAUC, ROCAUC, F2-score, and Cohen's kappa for different subsets of species, as well as an all-species mean. From this, it was determined that the random forest performed the best under our chosen metrics.
%We trained four different machine learning models on a data-set from iNaturalist containing coordinates and names of species identified at those coordinates. The four models used were a feed-forward neural network, random forest, logistic regression, and k-nearest neighbours. We evaluated the performance of these models on a test set by calculating the precision-recall area-under-curve, receiver operating characteristic area-under-curve, F2 score, and Cohen's kappa for different subsets of species in the test set, as well as an all-species mean. From this, it was determined that the random forest performed the best under our chosen metrics.
To improve the accuracy of our models we introduced six bio-climatic features and trained the KNN, FFNN and RF models on these features in addition to the given geospatial data. Overall, this led to some increases in F2 score, particularly for dense populations in the FFNN model. Using past and projected future temperature values, we determined the species that are most likely to be affected by climate change. We defined a "vulnerability score" and in conjunction with the improved eight-feature FFNN, was used to determine that the species in Table \ref{tab:vulnerability} are the most likely to be affected by climate change.

%by normalising the temperature anomaly across the Earth by the largest difference. This score, in conjunction with the improved eight-feature FFNN, was used to determine that the species in table\ref{tab:vulnerability} are the most likely in the given data-set to be affected by climate change.
